\documentclass[12pt]{article}
\usepackage[margin=1.33in]{geometry}
\usepackage[numbered,framed]{mcode}
\usepackage{graphicx}
\graphicspath{ {/Users/admin/Downloads/} }
\newcommand{\HRule}{\rule{\linewidth}{0.2mm}}
\newcommand{\Hrule}{\rule{\linewidth}{0.1mm}}

\makeatletter% since there's an at-sign (@) in the command name
\renewcommand{\@maketitle}{%
  \parindent=0pt% don't indent paragraphs in the title block
  \centering
  {\scshape \LARGE \@title}
  \HRule\par%
 \textit{\@author \hfill \@date}
  \par
}
\makeatother% resets the meaning of the at-sign (@)

\title{Investigating empirical convergence of the finite difference method for solving differential equations}
\author{Milosz Bolibrzuch}


\begin{document}
  \maketitle% prints the title block
  \vspace{0.6cm}
  
\section{Estimating convergence with known real value}
For this section, I will use the example of the equation $y'=xy$ and try to assess the convergence of the finite difference method through numerically estimating the solution to this equation. I use the following MATLAB code to do so:
\begin{lstlisting}
% 03-30:
% 
% This version of the solver focuses on estimating the convergence of a
% problem with a known solution, specifically y' = xy, y(0)=1. It
% accomplishes the following:
% 1. Plots the estimated solution of the equation.
% 2. Plots the convergence in the conventional, easy to read, logarithmic
% scale, for both the (forward difference NOTE: I'm still unsure about
% what to call the methods exactly) and (central difference). Analytic
% calculations indicate that the first method should have the order of
% convergence one, and the second order of convergence two.
% 3. Checks the empirical order of convergence with polyfit MATLAB
% function.

close all
clear all
n = 2;
max_m = 13;
m = 1;
error = zeros(1,max_m);
error1 = zeros(1,max_m);

while m <= max_m
    n = 2.^m;  
    h = 1/n;
    H(m) = h;
    y = zeros(1,n+2);
    x = zeros(1,n+2);
    real = zeros(1,n+2);
    y(1) = 1;
%     y(2) = exp(h.^2/2);
% We can use the result of the worse algorithm with one step as a
% second initial value. We can also use the derivative at the initial value
% to estimate the second value.
    y(2) = 1;
    x(1) = 1;
    x(2) = y(2);
    real(1) = 1;
    real(2) = y(2);
    for i = 1:n-2
        x(i+2) = x(i+1) + h*(i+1)*h*x(i+1);
        y(i+2) = y(i) + 2*y(i+1)*h*h*(i+1);
        real(i+2) = exp((((i+2)*h).^2)/2);
    end
    error1(m) = max(abs(x-real));
    error(m) = max(abs(y-real));
    m = m+1;
end
% 
% semilogy(error)
% hold on
% semilogy(error1)
% legend('Better', 'Worse');
% Check SUBPLOTS
% plot(y(1:n-2))
% hold on
% plot(real)
% Check for y'(x) = sin(xy) + y
% y(0) = 1
% Plot against exp(x^2/2-x), exp(x)
% This below is for conventional form
% Hflip = fliplr(H);
% errorflip = fliplr(error);
% errorflip1 = fliplr(error1);
% % plot(log2(Hflip), log2(errorflip))
% % hold on
% % plot(log2(Hflip), log2(errorflip1))
% % legend('Better' , 'Worse');
% % Plot a straight line to see if its perfectly straight
% Q = polyfit(log2(Hflip(1:12)), log2(errorflip(1:12)),1)
% R = polyfit(log2(Hflip(1:12)), log2(errorflip1(1:12)),1)
\end{lstlisting}
This is using the following relationship: $$f'(x) = \frac{f(x+h)-f(x-h)}{2h}+O(h^2).$$ As the above equation suggests, we know from basic analytic calculations that the convergence of this method should be of second order. 
First I will plot simply the error relation based on the number of iterations. 

\begin{figure}[h]
\caption{Convergence of error based on the amount of iterations}
\includegraphics[scale = 0.35]{thesis1pic}
\end{figure}

\newline Then I try to plot the same relationship in the loglog scale.

\begin{figure}[h]
\caption{Convergence of error based on the amount of iterations}
\includegraphics[scale = 0.35]{thesis1pic2}

\newline At this point, I am working to figure out why it isn't a straight line and how to adjust my method to make it a straight line. I am considering the fact that this is just iterating through n (amount of steps). Perhaps iterating through h (step size) could be a solution. I'm trying to adjust my code for that.
\section{Unknown solution}
The next step is to do the same (investigate convergence), but without assuming the knowledge of a solution. There are two main ways to proceed:
\newline 1. Compute a single reference value with a very small step size h and then proceed as if that reference value was a solution.
\newline 2. Look at the ratios of differences between approximations, usually for consecutively halved step size. It would be a series of elements looking roughly like the ratio below.
$$ \frac{u_{h}-u_{h/2}}{u_{h/2}-u_{h/4}}$$
\end{figure}
\end{document}